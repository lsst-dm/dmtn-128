\addtocounter{table}{-1}
\begin{longtable}{|p{0.145\textwidth}|p{0.8\textwidth}|}\hline
\textbf{Acronym} & \textbf{Description}  \\\hline

ADQL & Astronomical Data Query Language \\\hline
AURA & Association of Universities for Research in Astronomy \\\hline
AWS & Amazon Web Services \\\hline
Alert & A packet of information for each source detected with signal-to-noise ratio > 5 in a difference image during Prompt Processing, containing measurement and characterization parameters based on the past 12 months of LSST observations plus small cutouts of the single-visit, template, and difference images, distributed via the internet \\\hline
Alert Production & The principal component of Prompt Processing that processes and calibrates incoming images, performs Difference Image Analysis to identify DIASources and DIAObjects, packages and distributes the resulting Alerts, and runs the Moving Object Processing System \\\hline
BAC & Budget At Complete \\\hline
Broker & Software which receives and redistributes Alerts, and may also perform processing such as filtering for certain characteristics, cross-matching with non-LSST catalogs, and/or light-curve classification, in order to identify and prioritize targets for follow-up and/or make scientific analyses.  \\\hline
Butler & A middleware component for persisting and retrieving image datasets (raw or processed), calibration reference data, and catalogs \\\hline
CC & Change Control \\\hline
CCOB & Camera Calibration Optical Bench \\\hline
Camera & The LSST subsystem responsible for the 3.2-gigapixel LSST camera, which will take more than 800 panoramic images of the sky every night. SLAC leads a consortium of Department of Energy laboratories to design and build the camera sensors, optics, electronics, cryostat, filters and filter exchange mechanism, and camera control system \\\hline
Center & An entity managed by AURA that is responsible for execution of a federally funded project \\\hline
DM & Data Management \\\hline
DMCCB & DM Change Control Board \\\hline
DMTN & DM Technical Note \\\hline
DMTR & DM Test Report \\\hline
DRP & Data Release Production \\\hline
Data Access Center & Part of the LSST Data Management System, the US and Chilean DACs will provide authorized access to the released LSST data products, software such as the Science Platform, and computational resources for data analysis. The US DAC also includes a service for distributing bulk data on daily and annual (Data Release) timescales to partner institutions, collaborations, and LSST Education and Public Outreach (EPO).  \\\hline
Data Management & The LSST Subsystem responsible for the Data Management System (DMS), which will capture, store, catalog, and serve the LSST dataset to the scientific community and public. The DM team is responsible for the DMS architecture, applications, middleware, infrastructure, algorithms, and Observatory Network Design. DM is a distributed team working at LSST and partner institutions, with the DM Subsystem Manager located at LSST headquarters in Tucson \\\hline
Data Release Production & An episode of (re)processing all of the accumulated LSST images, during which all output DR data products are generated. These episodes are planned to occur annually during the LSST survey, and the processing will be executed at the Archive Center. This includes Difference Imaging Analysis, generating deep Coadd Images, Source detection and association, creating Object and Solar System Object catalogs, and related metadata \\\hline
Document & Any object (in any application supported by DocuShare or design archives such as PDMWorks or GIT) that supports project management or records milestones and deliverables of the LSST Project \\\hline
FGCM &  Forward Global Calibration Model \\\hline
FTE & Full Time Equivalent \\\hline
HSC & Hyper Suprime-Cam \\\hline
IVOA & International Virtual-Observatory Alliance \\\hline
LATISS & LSST Atmospheric Transmission Imager and Slitless Spectrograph \\\hline
LDM & LSST Data Management (Document Handle) \\\hline
LSST & Large Synoptic Survey Telescope \\\hline
NCSA & National Center for Supercomputing Applications \\\hline
OPS & Operations \\\hline
Offer & A response to a solicitation that, if accepted, would bind the offeror to perform the work described in resultant contract. Responses to sealed bidding are offers that are often referred to as 'bids' or 'sealed bids;' responses to a request for proposals (RFP, negotiated-type procurements) are offers often referred to as 'proposals' responses to a request for quotations (RFQ) are not offers and are generally called 'quotes' \\\hline
Operations & The 10-year period following construction and commissioning during which the LSST Observatory conducts its survey \\\hline
Operations Rehearsal & A data management system prototype project employing the same methods, tools, personnel, and technologies as the real system in order to introduce and validate new algorithms, functionality, and infrastructure. Previously referred to as a data challenge \\\hline
PCW & Project Community Workshop \\\hline
PSF & Point Spread Function \\\hline
Project Manager & The person responsible for exercising leadership and oversight over the entire LSST project; he or she controls schedule, budget, and all contingency funds \\\hline
Prompt Processing & The processing that occurs at the Archive Center on the nightly stream of raw images coming from the telescope, including Difference Imaging Analysis, Alert Production, and the Moving Object Processing System. This processing generates Prompt Data Products \\\hline
Qserv & Proprietary Database built by SLAC for LSST \\\hline
Release & Publication of a new iteration of an existing document following approval of changes through the change control process. Upon release, the new iteration becomes the current baseline and the preferred version in the archive \\\hline
Review & Programmatic and/or technical audits of a given component of the project, where a preferably independent committee advises further project decisions, based on the current status and their evaluation of it. The reviews assess technical performance and maturity, as well as the compliance of the design and end product with the stated requirements and interfaces \\\hline
SODA & Server-side Operations for Data Access \\\hline
SPIE & The international society for optics and photonics \\\hline
SQuaRE & Science Quality and Reliability Engineering \\\hline
SQuaSH & Science Quality Analysis Harness \\\hline
SUIT & Science User Interface and Tools \\\hline
Science Pipelines & The library of software components and the algorithms and processing pipelines assembled from them that are being developed by DM to generate science-ready data products from LSST images. The Pipelines may be executed at scale as part of LSST Prompt or Data Release processing, or pieces of them may be used in a standalone mode or executed through the LSST Science Platform. The Science Pipelines are one component of the LSST Software Stack \\\hline
Science Platform & A set of integrated web applications and services deployed at the LSST Data Access Centers (DACs) through which the scientific community will access, visualize, and perform next-to-the-data analysis of the LSST data products \\\hline
Specification & One or more performance parameter(s) being established by a requirement that the delivered system or subsystem must meet \\\hline
Stripe 82 & A 2.5° wide equatorial band of sky covering roughly 300 square degrees that was observed repeatedly in 5 passbands during the course of the SDSS, In part for calibration purposes \\\hline
TAP & Table Access Protocol \\\hline
TB & TeraByte \\\hline
TCAM & Technical Control (or Cost) Account Manager \\\hline
US & United States \\\hline
VCD & Verification Control Document \\\hline
Validation & A process of confirming that the delivered system will provide its desired functionality; overall, a validation process includes the evaluation, integration, and test activities carried out at the system level to ensure that the final developed system satisfies the intent and performance of that system in operations \\\hline
Verification & The process of evaluating the design, including hardware and software - to ensure the requirements have been met;  verification (of requirements) is performed by test, analysis, inspection, and/or demonstration \\\hline
WISE & Wide-field Survey Explorer \\\hline
brighter-fatter effect & The common term used to refer to one of the photometric qualities of the LSST camera: sources with a higher flux have a broader PSF. This is accounted for during calibration \\\hline
calibration & The process of translating signals produced by a measuring instrument such as a telescope and camera into physical units such as flux, which are used for scientific analysis. Calibration removes most of the contributions to the signal from environmental and instrumental factors, such that only the astronomical component remains \\\hline
pipeline & A configured sequence of software tasks (Stages) to process data and generate data products. Example: Association Pipeline \\\hline
\end{longtable}
